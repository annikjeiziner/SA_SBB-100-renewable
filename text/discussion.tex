\chapter{Discussion}
Simulating the model without any share in renewables show, that the model approximates reality quite well: While the import power peaks are too high and the water levels do not fully correspond to the real pattern, the general power dispatch over one year and the energy dispatch fit reality according to the experts at SBB quite well. To decrease the power peaks one could introduce an availability factor of 70\% leading to a decreased maximum possible power import at one time. For the water levels more boundary conditions should be formulated to force the model to follow the real patterns. For example there could be a condition that the lakes shall be empty in May and full in October. 

The results including a share of the renewables solar and wind depend heavily on the tuning of the weighting factors of $W_1, W_2$ and $W_3$.
Most of the simulations were done with the weighting factors of $W_1 = W_2 = 1$ and $W_3 = 10^{-20}$, because this approximated the goal of the model best. With this setting, the model focused on minimizing the curtailment of the renewables instead of forcing the exchange cost to be low taking into account high curtailments. This was mainly done due to the fact that the model considers no investment costs. With weighting factors $W_1 = W_2 = W_3 = 1$ the model would invest a huge amount in solar or wind, just to reduce the exchange cost and because this investments do not cost anything. This is clearly not the realistic case. Therefore, the weighting factor for cost was chosen to be extremely small, such that the model would focus on the curtailment of renewables. It has to be added, that the exchange cost was only implemented to introduce a seasonal pattern to the model and a certain limit to the exchange. 

However, in the end the weighting factors stay a tuning factor. One can attribute the importance individually to its needs. Changing the weighting factors leads to different results. In this thesis only the extreme weighting factors were and their effects were shown. 

Nevertheless, whenever the model is free to chose the renewable energy source, it prefers wind. This is due to the fact that wind power is almost continuously available. Therefore it is also available in the morning and evening and during winter times. These are the times when SBB needs the most energy and these are also the times when prices are high. 

Solar power however is rarely chosen by the model. Only in the scenarios with solar location in the alpine regions or with a future load profile the model decides in favor of some share of solar. On the one hand side this is due to the fact that in alpine regions the global radiation is higher during winter months and therefore more solar power is available during high demand periods. On the other side a future load profile has a higher demand in general and therefore also at lunchtimes. Here, solar can provide a part of the demanded energy. 
