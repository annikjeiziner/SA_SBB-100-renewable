
%****Header*****

%%%in here are the most important definitions of this template
%%%please only make changes if you are sure that you need them.
%scrreport is part of KOMA script: https://www.ctan.org/pkg/koma-script?lang=en
%if something is not to your liking or you want a new feature, make sure that 
% 1) KOMA doesnt already provide the feature
% 2) the package is KOMA compatible!

\documentclass[
fleqn,											
fontsize=11pt,
paper=a4,
twoside,open=right,					%use this option for printing on both sides!
bibliography=totoc,listof=totoc,BCOR=25mm,
%ngerman  %use only if you write in German!
english,
DIV = 12 %decides how full the pages are
]{scrreprt}
%% A. Ritter, 12.1.2017 -> changed BCOR from 5 to 25 and DIV from 11 to 12 to make it more appealing when printed

%%%%%%%%%%%%%%%%%%%%%%%%%%%%%%%%%%%%%%%%%%%
% text and language packages
%%%%%%%%%%%%%%%%%%%%%%%%%%%%%%%%%%%%%%%%%%%

%%% it's a good idea to know what each of the following packages does
%%% if you dont know -> GOOGLE it, we selected only helpful ones.

\usepackage[utf8]{inputenc} %encoding of .tex files -> please use utf8 exclusively!
\usepackage[T1]{fontenc} %encoding of the fonts, T1 should work best
\usepackage[english]{babel} %use english hyphenation and such
\usepackage{here} 

\usepackage{ifpdf} %necessary package to distinguish between pdfLatex and latex->ps compilations
\usepackage{graphicx} %provides \includegraphics and such
%%the following couple of packages introduce capability to deal with .eps files
\usepackage{epstopdf} %adds .eps capability to \includegraphics but doesn't do anything with psfrag
\ifpdf %set compression of bitmap files which are saved as .eps so that they don't become worse when making a .pdf
\epstopdfsetup{update,suffix=-generated2}
\epstopdfDeclareGraphicsRule{.eps}{pdf}{.pdf}{epstopdf --gsopt="-dPDFSETTINGS=/prepress -dAutoFilterColorImages=false -dAutoFilterGrayImages=false -sColorImageFilter=FlateEncode -sGrayImageFilter=FlateEncode -sCompressPages=false -dPreserveHalftoneInfo=true" #1 --outfile=\OutputFile}
\fi
\usepackage[crop=pdfcrop,mode=nonstop]{pstool} %adds psfrag capability so that .eps figures can
%%

\usepackage{rotating} %rotate figures and tables
\usepackage{frame} %put frames around things
\usepackage{amstext,amssymb,amsbsy,amsmath} %add a lot of math capability
\usepackage{adjustbox} %scales floats such as tables and figures to fit on certain size, i.a. pagewidth
\usepackage{subcaption} %provides elegant method for adding subfigures, subplots, subcaptions 
\usepackage{textcomp} %provides degree symbols and some other useful things, can be achieved with siunitx too

%%you can draw electric circuit diagrams
%%as well as plot data pretty with tikz (circuittikz for the former, pgfplots for the latter)
%
\usepackage[european]{circuitikz} %draw electric diagrams
%\usepackage{0_packages/circuitikz/tex/circuitikz} %draw electric diagrams
\usepackage{tikz} %draw a lot of stuff
\usepackage{tikzscale}
\usetikzlibrary{shapes,arrows}
\usepackage{xstring}
\usepackage{pgfplots} %plot symbolic functions and data directly in latex via tikz
\usepgfplotslibrary{colorbrewer} %have distinct and pleasing colors in pgfplots
\usetikzlibrary{colorbrewer}
\pgfplotsset{compat=newest}
\newcommand*{\equal}{=}
\usepackage{tkz-euclide}
\usetkzobj{all}
\usepackage{cancel}

\usepackage{siunitx} %elegantly handle numbers and their units

\usepackage{pdfpages} %if you need to include entire pages as pdf (mostly for the appendix)

\usepackage{enumitem} %more enumerable environments
\usepackage[hidelinks]{hyperref} %makes all kinds of links (references, table of contents, etc) clickable
\usepackage[section]{placeins} %assures that all floats are in the section they should be in before starting a new one

%%provides you with \todo{} and \missingfigure and such: 
%%http://mirror.switch.ch/ftp/mirror/tex/macros/latex/contrib/todonotes/todonotes.pdf
\usepackage{todonotes} % does the todo list. should be loaded after other packages. 

%all of the numbers and units are done with siunitx, please check out
%http://www.pirbot.com/mirrors/ctan/macros/latex/contrib/siunitx/siunitx.pdf
\usepackage{siunitx}

%\usepackage{cite} %does references and bibliography
\usepackage[backend=biber, style=ieee, sorting=none]{biblatex}
\addbibresource{literature.bib}

\pagestyle{headings} 				%schaltet die Kopfzeile ein. Die Option {headings} erzeugt die Nummer und den Namen des Kapitels

\setcounter{tocdepth}{1} %make table of content 1-deep (i.e. chapters + sections but not subsections)
%
%%%%%%%%%%%%%%%%%%%%%%%%%%%%%%%%%%%%%%%%%%%%
%%Fuss- und Kopfzeilendefinition
%%%%%%%%%%%%%%%%%%%%%%%%%%%%%%%%%%%%%%%%%%%%
\usepackage{fancyhdr}
\fancyhead{}
\fancyfoot{}
\renewcommand{\headrulewidth}{0.5pt}

%%%% double sided layout: use this
\fancyhead[EL, OR]{ \thepage}
\fancyhead[EC]{\textsl{\leftmark}}
\fancyhead[OC]{\textsl{\rightmark}}

%%%% single sided layout: use this
%\fancyhead[R]{ \thepage}
%\fancyhead[C]{\textsl{\leftmark}}

\pagestyle{fancy}

\renewcommand{\chaptermark}[1]{%
\markboth{\thechapter\ #1}{}}

\renewcommand{\sectionmark}[1]{%
\markright{\thesection\ #1}}

%%%%%%%%%%%%%%%%%%%%%%%%%%%%%%%%%%%%%%%%%%%%
%%Glossar und Abkürzungen
%%%%%%%%%%%%%%%%%%%%%%%%%%%%%%%%%%%%%%%%%%%%
\usepackage[acronym, toc]{glossaries}
\makeglossaries
\input{text/glossary.tex}

%%%%%%%%%%%%%%%%%%%%%%%%%%%%%%%%%%%%%%%%%%%%
%%Aufzählung von Parametern / Variabeln
%%%%%%%%%%%%%%%%%%%%%%%%%%%%%%%%%%%%%%%%%%%%
%\newenvironment{conditions}
%  {\par\vspace{\abovedisplayskip}\noindent\begin{tabular}{>{$}l<{$} @{${}={}$} l}}
%  {\end{tabular}\par\vspace{\belowdisplayskip}}
  
%%%%%%%%%%%%%%%%%%%%%%%%%%%%%%%%%%%%%%%%%%%%
%%zentrierte Gleichungen
%%%%%%%%%%%%%%%%%%%%%%%%%%%%%%%%%%%%%%%%%%%%
%\usepackage[fleqn]{amsmath}

%%%%%%%%%%%%%%%%%%%%%%%%%%%%%%%%%%%%%%%%%%%%
%% um Tabelle in Textbreite zu behalten. 
%%%%%%%%%%%%%%%%%%%%%%%%%%%%%%%%%%%%%%%%%%%%
\usepackage{array}
\newcolumntype{L}[1]{>{\raggedright\let\newline\\\arraybackslash\hspace{0pt}}m{#1}}
\newcolumntype{C}[1]{>{\centering\let\newline\\\arraybackslash\hspace{0pt}}m{#1}}
\newcolumntype{R}[1]{>{\raggedleft\let\newline\\\arraybackslash\hspace{0pt}}m{#1}}

%%%%%%%%%%%%%%%%%%%%%%%%%%%%%%%%%%%%%%%%%%%%
%% To have landscape mode (for big tables and figures)
%%%%%%%%%%%%%%%%%%%%%%%%%%%%%%%%%%%%%%%%%%%%
\usepackage{lscape}


\usepackage{caption}
\usepackage{subcaption}