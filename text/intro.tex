\chapter{Introduction}
The Swiss Federal Railway System is one of the densest rail networks in the world \cite{uic}. To provide the trains with energy the Schweizerische Bundesbahnen (SBB) operates a separate power grid at 16.7\,Hz\,AC. Most of the energy is generated by hydro power plants in Switzerland which are partly or fully owned by SBB. However, SBB also relies on the power exchange with the 50\,Hz grid. For this purpose they own several rotating and static frequency converters \cite{sbb1}. 

The energy obtained from the 50\,Hz grid is mostly generated by hydro power plants and nuclear power plants \cite{BFS}. 

Having the exchange with the 50\,Hz grid, SBB therefore has in total a renewable energy share of SBB of 90\%. SBB has however set the ambitious target of 100\% renewable energy generation until 2025 \cite{sbb2}. 

To achieve this target SBB launched a project called "rENewable". The goals of this project include to prove that renewable energies such as solar and wind power can be directly feed into the 16.7\,Hz grid for traction purposes. Another goal of this project is to guarantee an ideal supplementary portfolio with renewable energies such as solar and wind according to SBB's load profile \cite{rENewable}. 

In this thesis, a Generation Expansion Planning (GEP) based on an energy balance approach to determine the optimal type and share for renewable solar and wind generation investments is done. In the first section all mathematical formulations leading to the optimization problem will be elaborated. Afterwards a case study and the results obtained from the simulations will be shown. Finally, recommendations will be presented in the discussion, which is then followed by a conclusion. 