\chapter{Conclusion}
A optimization problem to derive the optimal type (solar or wind) without considering specific locations and investment costs is introduced in this thesis. Several scenarios were simulated with this model and the results were presented. 

The results clearly show a clear favoring of wind power due to its continuous availability, also during high load and high price periods, were solar power is less available. 

By changing the weighting factors, the results change significantly. In the next steps it is therefore to decide on the weighting factor that corresponds to SBBs needs and simulate the corresponding results. 

The future work however should focus on the investment costs of solar and wind power. Having this included, the renewable target can be formulated as an inequality constraint instead of a strict equality constraint as it is now. The model would therefore have an intrinsic incentive to keep the investments low. 

Furtermore, more boundary conditions on the water levels and an availability factor to reduce the maximum possible import power should be introduced.  

Further steps could also consider the locations of solar and wind power in more detail. 